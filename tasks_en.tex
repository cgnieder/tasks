% arara: pdflatex
% arara: biber
% arara: pdflatex
% arara: pdflatex
% --------------------------------------------------------------------------
% the TASKS package
% 
%   horizontal columned lists
% 
% --------------------------------------------------------------------------
% Clemens Niederberger
% Web:    https://bitbucket.org/cgnieder/exsheets/
% E-Mail: contact@mychemistry.eu
% --------------------------------------------------------------------------
% Copyright 2011-2014 Clemens Niederberger
% 
% This work may be distributed and/or modified under the
% conditions of the LaTeX Project Public License, either version 1.3
% of this license or (at your option) any later version.
% The latest version of this license is in
%   http://www.latex-project.org/lppl.txt
% and version 1.3 or later is part of all distributions of LaTeX
% version 2005/12/01 or later.
% 
% This work has the LPPL maintenance status `maintained'.
% 
% The Current Maintainer of this work is Clemens Niederberger.
% --------------------------------------------------------------------------
% The tasks package is part of the exsheets bundle
% --------------------------------------------------------------------------
% If you have any ideas, questions, suggestions or bugs to report, please
% feel free to contact me.
% --------------------------------------------------------------------------
\documentclass[load-preamble+]{cnltx-doc}
\usepackage[more]{tasks}

\setcnltx{
  package  = {tasks} ,
  authors  = Clemens Niederberger ,
  email    = {contact@mychemistry.eu} ,
  url      = {https://bitbucket.org/cgnieder/exsheets/} ,
  info     = {%
    {\small part of the \href{exsheets_en.pdf}{\ExSheets}
      bundle}\\[\baselineskip]
    create horizontal columned lists%
  } ,
  add-cmds = {
    checkedchoicebox ,
    choicebox,
    NewTasks,
    settasks,
    startnewitemline ,
    task
  } ,
  add-silent-cmds = {
    choice, correct,
    DeclareInstance, DeclareTemplateInterface,
    leftthumbsup,
    s, sample
  }
}

\newpackagename\ExSheets{ExSheets}
\newpackagename\ExSheetslistings{ExSheets-listings}
\newpackagename\cntformats{cntformats}
\newpackagename\Tasks{tasks}

% ----------------------------------------------------------------------------
% other packages, bibliography, index
\usepackage{xcoffins,tikz,wasysym,enumitem,booktabs,siunitx}
\usepackage[accsupp]{acro}
\DeclareAcronym{id}{
  short     = id ,
  long      = Identifier ,
  format    = \scshape ,
  pdfstring = ID ,
  accsupp   = ID
}

\usepackage{filecontents}
\usepackage{csquotes}



% ----------------------------------------------------------------------------
% example definitions that have to be done in the preamble:
\usepackage{exsheets}
\usepackage{dingbat}
\NewTasks[style=multiplechoice]{multiplechoice}[\choice](3)
\newcommand*\correct{\PrintSolutionsTF{\checkedchoicebox}{\choicebox}}


\def\s{This is some sample text we will use to create a somewhat
  longer text spanning a few lines.}
\def\sample{\s\ \s\par\s}

\begin{document}

\section{Motivation}
Originally \Tasks\ has been an integral part of \ExSheets\changedversion{0.7}.
However, users told me that it indeed could be useful to have it as a
stand-alone package not having to load the whole \ExSheets\ beast just for
having the \env{tasks} environment available.  Since I agree with this the
environment has been extracted into a package if its own, \Tasks.

The reason for the \env{tasks} environment is an unwritten agreement in German
maths textbooks (in (junior) high school, especially) to organize exercises in
columns counting horizontally rather than vertically.  That is what
\code{tasks} primarily is for. If you don't need this feature you're better
off using traditional \LaTeX{} lists and the \pkg{enumitem} package for
customization.

\section{License and Requirements}\label{sec:license}
\license

\Tasks\ requires the \bnd{l3kernel}~\cite{bnd:l3kernel} bundle
,\needpackage{xparse}, \pkg{xtemplate} and \needpackage{l3keys2e} which are
part of the \bnd{l3packages}~\cite{bnd:l3packages} bundle,
\pkg{epic}~\cite{pkg:epic}, \pkg{cntformats}\footnote{Part of the \ExSheets\
  bundle}, and \pkg{environ}~\cite{pkg:environ}.


\section{How it works}
\subsection{The Basics}
The \env{tasks} environment is similar to a list like \env{enumerate} but not
the same.  Here are some of the differences:
\begin{itemize}
  \item A first difference: there is no pagebreak possible inside an item but
    only between items.
  \item A second difference: the enumeration default is a), b), c) \ldots
  \item A third difference: there is a split at \emph{every} occurrence of the
    item separator.  For this reason the default separator is not \cs*{item}
    but \cs{task} so it is unique to this environment only.
  \item A fourth difference: the \env{tasks} environment cannot be nested.
    You can, however, use an \env*{itemize} environment or something in it.
  \item A fifth difference: verbatim material cannot be used in it.  You'll
    have to use \cs*{string}, \cs*{texttt} or \cs*{detokenize}.  If this
    won't suffice then don't use \env{tasks}.
%  \item A sixth difference: %footnotes
\end{itemize}

\begin{environments}
  \environment{tasks}[\oarg{options}\darg{num of columns}]
    List like environment where the single items are introduced with
    \cs{task}.
\end{environments}
Let's see an example:
\begin{example}
  % \sample is defined to contain some sample text:
  % \def\s{This is some sample text we will use to create a somewhat
  %   longer text spanning a few lines.}
  % \def\sample{\s\ \s\par\s}
  Some text before the list.
  \begin{tasks}
    \task \sample
    \task \sample
    \task \sample
  \end{tasks}
  And also some text after it.
\end{example}

The environment takes the optional argument \darg{num of columns} with which
the number of columns used by the environment is specified.
\begin{example}
  \begin{tasks}(2)
    \task \sample
    \task \s\ \s
    \task \s
    \task \sample
    \task \s\par\s
  \end{tasks}
\end{example}

\subsection{Introducing a New Row}
Sometimes it may come in handy if the current\sinceversion{0.9} row of items
could be terminated and a new one is started.  This is possible with the
following command:
\begin{commands}
  \command{startnewitemline}
    Introduce a new line in a \env{tasks} environment.
\end{commands}
\begin{example}
  \begin{tasks}(4)
    \task the first
    \task the second
    \task the third
    \task the fourth
    \task \rlap{the fifth item is way too long for this so we start a new row}
      \startnewitemline 
    \task the sixth
    \task the seventh
    \task \rlap{the eighth item also is too long} \startnewitemline
    \task the nineth
    \task the tenth
  \end{tasks}
\end{example}



\section{Available Options}\label{sec:tasks:options}
The \Tasks\ package has one package option which also is called when you load
\ExSheets\ with the \option{load-tasks} option.
\begin{options}
  \opt{more}
    Load additional instances for the \env{tasks} object, details are
    explained later in section~\ref{sec:tasks:instances}.
\end{options}

The environment itself has some more options, namely the following ones that
can be set using a setup command:
\begin{commands}
  \command{settasks}[\marg{options}]
    Setup command for \Tasks.
\end{commands}
\begin{options}
  \keyval{style}{instance}\Default
     Choose the instance to be used.  Read more on this in
     section~\ref{sec:tasks}.
  \keyval{counter-format}{counter specs}\Default
    \sinceversion{0.9}Sets a custom label.  The letters \code{tsk} are
    replaced with the task-counter.  An optional argument directly following
    these letters specifies the counter format: \code{1}: \cs*{arabic},
    \code{a}: \cs*{alph}, \code{A}: \cs*{Alph}, \code{r}: \cs*{roman} and
    \code{R}: \cs*{Roman}.
  \keyval{label-format}{code}\Default
    \changedversion{0.9}Can be used to apply a formatting like, \eg,
    \cs*{bfseries} to the labels.
  \keyval{label}{code}\Default
    \changedversion{0.9}Overwrite the automatic label to a custom one.
  \keyval{label-width}{dim}\Default{1em}
    Sets the width of the item labels.
  \keyval{label-offset}{dim}\Default{.3333em}
    \sinceversion{0.7}Sets the offset, \ie, the distance between label and
    item.
  \keyval{item-indent}{dim}\Default{2.5em}
    \sinceversion{0.9a}The indent of an item, \ie, the horizontal space
    available for both label and label-offset.  If
    \[
      \text{\code{indent}} =
      \text{\code{label-width}} + \text{\code{label=offset}}
    \]
    the label will align with the textblock above (if
    \keyis{label-align}{left} is set).  Please see figure~\ref{fig:lengths}
    for a sketch of the available lengths and how they are set.
  \keyval{column-sep}{dim}\Default{0pt}
    A horizontal length that is inserted between columns ot items.
  \keychoice{label-align}{left,right,center}\Default{left}
    \sinceversion{0.7}Determines how the labels are aligned within the
    label-box whose width is set with \option{label-width}.
  \keyval{before-skip}{skip}\Default{0pt}
    Sets the skip before the list.
  \keyval{after-skip}{skip}\Default{0pt}
    Sets the skip after the list.
  \keyval{after-item-skip}{skip}\Default{1ex plus 1ex minus 1ex}
    \sinceversion{0.9}This vertical skip is inserted between rows of items.
  \keybool{resume}\Default{false}
    The enumeration will resume from a previous \env{tasks} environment.  In
    order to use this option properly you shouldn't mix different \env{tasks}
    environments that both count their items.
  \keybool{debug}\Default{false}
    If set to true \cs*{fboxsep} is set to \code{0pt} inside the \env{tasks}
    environment and \cs*{fbox} is used to draw a frame around the label boxes
    and the item boxes.
\end{options}

\begin{figure}
  \centering
  \begin{tikzpicture}[every node/.style={font=\footnotesize},scale=.5]
    \coordinate (itemedge1) at (2,2) ;
    \coordinate (itemedge2) at (13,2) ;
    \draw
      (itemedge1) -- ++(8,0) -- ++(0,-2) -- ++(-8,0) -- cycle ;
    \draw
      (itemedge1) ++(-.5,0) coordinate(labeledge1)
      -- ++(-1,0) --++ (0,-1) --++(1,0) --++(0,1) ;
    \draw (itemedge1) ++(-2,0) -- ++(0,-2) ;
    \draw
      (itemedge2) -- ++(8,0) -- ++(0,-2) -- ++(-8,0) -- cycle ;
    \draw
      (itemedge2) ++(-.3,0) coordinate(labeledge2)
      -- ++(-1,0) --++ (0,-1) --++(1,0) --++(0,1) ;
    \draw (itemedge2) ++(-2,0) -- ++(0,-2) ;
    \draw[<->] (itemedge2) ++(-2,0) --node[above]{column sep} ++(-1,0) ;
    \draw[<->] (0,-.5) --node[below]{item indent} (2,-.5) ;
    \draw[<->] (2,-.5) --node[below]{item width} (10,-.5) ;
    \draw[<->] (labeledge1) ++(0,1) --node[above]{label width} ++(-1,0) ;
    \draw[<->] (labeledge1) --node[above]{item offset} ++(.5,0) ;
  \end{tikzpicture}
  \caption{A visual representation of the used lengths.}
  \label{fig:lengths}
\end{figure}

Now the same list as above but with three columns and a different label:
\begin{example}
  \begin{tasks}[counter-format=(tsk[r]),label-width=4ex](2)
    \task \sample
    \task \s\ \s
    \task \s
    \task \sample
    \task \s\par\s
  \end{tasks}
\end{example}
% \begin{tasks}[counter-format=(tsk[r]),label-width=4ex](3)
%  \task \sample
%  \task \s\ \s
%  \task \s
%  \task \sample
%  \task \s\par\s
% \end{tasks}

Let's use it inside a question, \ie, inside \ExSheets' \env{question}
environment:
\begin{example}
  % since settings are local the following ones will be lost
  % outside this example;
  \settasks{
    counter-format = qu.tsk ,
    item-indent    = 2em ,
    label-width    = 2em ,
    label-offset   = 0pt
  }
  \begin{question}[type=exam]{4}
    I have these two tasks for you. Shall we begin?
    \begin{tasks}(2)
      \task The first task: easy!
      \task The second task: even more so!
    \end{tasks}
  \end{question}
  \begin{solution}[print]
    Now, let's see\ldots\ ah, yes:
    \begin{tasks}
      \task This is the first solution. Told you it was easy.
      \task This is the second solution. And of course you knew that!
    \end{tasks}
  \end{solution}
\end{example}

Finally let's see what the \option{debug} option does:
\begin{example}
  \settasks{debug}
  \begin{tasks}(2)
    \task \sample
    \task \sample
  \end{tasks}
\end{example}

\section{Available Instances}\label{sec:tasks:instances}
When you use the package option \option{more} of the \Tasks\ package or load
\ExSheets\ with the \option{load-tasks} option there are currently three
additional instances for the \code{tasks} object available:
\begin{description}
  \item[itemize] uses \cs*{labelitemi} as labels.
  \item[enumerate] enumerates the items with 1., 2., \ldots
  \item[multiplechoice] a --~well~-- `multiple choice' list.
\end{description}
\begin{example}
  \begin{tasks}[style=itemize](2)
    \task that's just how\ldots
    \task \ldots we expected
  \end{tasks}
  \begin{tasks}[style=enumerate](2)
    \task that's just how\ldots
    \task \ldots we expected
  \end{tasks}
  \begin{tasks}[style=multiplechoice](2)
    \task that's just how\ldots
    \task \ldots we expected
  \end{tasks}
\end{example}

\section{Custom Labels}
If you want to change a single label inside a list, you can use the optional
argument of \cs{task}. This will temporarily overwrite the default label.
\begin{example}[side-by-side]
  \begin{tasks}[style=itemize]
    \task a standard item
    \task another one
    \task[+] a different one
    \task and another one
  \end{tasks}
\end{example}

\section{New Tasks}
It is possible to add custom environments that work like the \code{tasks}
environment.
\begin{commands}
  \command{NewTasks}[\oarg{options}\marg{name}\oarg{separator}\darg{cols}]
    Define environment \meta{name} that uses \meta{separator} to introduce a
    new item.  Default for \meta{separator} is \cs{task}, default for
    \meta{cols} is \code{1}.  The \meta{options} are the ones described in
    section~\ref{sec:tasks:options}.
  \command{RenewTasks}[\oarg{options}\marg{name}\oarg{separator}\darg{cols}]
    Renew environment previously defined with \cs{NewTasks}.
\end{commands}
The \env{tasks} environment is defined as follows:
\begin{sourcecode}
  \NewTasks{tasks}
\end{sourcecode}

The separator does not have to be a control sequence:
\begin{example}
  % preamble:
  % \usepackage{dingbat}
  \NewTasks[label=\footnotesize\leftthumbsup,label-width=1.3em]{done}[*]
  \begin{done}
    * First task
    * Second Task
  \end{done}
\end{example}
Although this might seem handy or even nice I strongly advice against using
something different than a command sequence. Remember that the items will be
split at \emph{every} occurrence of the separator.  So in order to use the
separator (here for example for a starred variant of a command) within an item
it has to be hidden in braces.  This is avoided of you use a command sequence
which even doesn't have to be defined.

Let's say you want a \env*{multiplechoice} environment that has three columns
in its default state.  You could do something like this:
\begin{example}
  % preamble:
  % \NewTasks[style=multiplechoice]{multiplechoice}[\choice](3)
  % \newcommand*\correct{\PrintSolutionsTF{\checkedchoicebox}{\choicebox}}
  %
  % \PrintSolutionsTF and the {question} environment are provided
  % by the ExSheets package
  \begin{question}
    \begin{multiplechoice}
      \choice First choice
      \choice Second choice
      \choice[\correct] Third choice
    \end{multiplechoice}
  \end{question}
  \begin{solution}[print]
    \begin{multiplechoice}
      \choice First choice
      \choice Second choice
      \choice[\correct] Third choice
    \end{multiplechoice}
  \end{solution}
\end{example}

The last example shows you two additional commands:
\begin{commands}
  \command{choicebox}[\quad\choicebox]
    Print an empty square.
  \command{checkedchoicebox}[\quad\checkedchoicebox]
    Print a crossed-out square.
\end{commands}


\section{Styling \Tasks}
Equivalent to the styling of \ExSheets\ \Tasks\ uses \pkg{xtemplate} to
declare additional instances for the lists.

\subsection{The \code{tasks} Object}\label{sec:tasks}
The object that's defined by \Tasks\ is the `tasks' object.  This time there
are four instances available for the one template (again `default') that was
defined.

\subsubsection{Available Options}
This section only lists the options that can be used when defining an instance
of the `default' template.  The following subsections will give some examples
of their usage.

\begin{sourcecode}
  \DeclareTemplateInterface{tasks}{default}{3}
    {
      % option        : type      = default
      enumerate       : boolean   = true    ,
      label           : tokenlist           ,
      indent          : length    = 2.5em   ,
      counter-format  : tokenlist = tsk[a]) ,
      label-format    : tokenlist           ,
      label-width     : length    = 1em     ,
      label-offset    : length    = .3333em ,
      after-item-skip : skip      = 1ex plus 1ex minus 1ex
    }
\end{sourcecode}

\subsubsection{Predefined Instances}
This is rather brief this time:
\begin{sourcecode}
  % ALPHABETIZE: a) b) c)
  \DeclareInstance{tasks}{alphabetize}{default}{}
  % available when `load-tasks=true':
  % ITEMIZE:
  \DeclareInstance{tasks}{itemize}{default}
    {
      enumerate   = false ,
      label-width = 1.125em
    }
  % ENUMERATE:
  \DeclareInstance{tasks}{enumerate}{default}
    { counter-format = tsk. }
  % MULTIPLECHOICE:
  \DeclareInstance{tasks}{multiplechoice}{default}
    {
      enumerate = false       ,
      label     = \choicebox  ,
    }
\end{sourcecode}

\end{document}
